\documentclass[english,11pt,twoside]{article}

\usepackage[utf8]{inputenc}
\usepackage{ifthen}
\usepackage[T1]{fontenc}
\usepackage[english]{babel}
\usepackage{lmodern}
\usepackage[a4paper]{geometry}
\geometry{tmargin=2.4cm,bmargin=2.4cm,innermargin=2cm,outermargin=2cm}
\usepackage{fancyhdr}
\usepackage{csquotes}
\usepackage[squaren,Gray]{SIunits}
\usepackage{graphicx}
\usepackage{booktabs}
\usepackage{xargs}
\usepackage[hidelinks,hyperfootnotes=false]{hyperref}
\usepackage{nicefrac}
\usepackage{amssymb}


\newcommand*{\nom}[1]{\textnormal{\textsc{#1}}}
\newcommand{\titre}[1]{%
\title{#1}}

\newcommand{\ingredientname}{Ingredients}
\newcommand{\quantityname}{Quantity}
\newcounter{step}
\setcounter{step}{0}
\newcommand{\step}{\stepcounter{step}\vspace{0.5cm}\par\noindent\textit{Step \thestep} --  }

\newcommand{\partie}[1]{
	\addcontentsline{toc}{section}{#1}
}

\newenvironmentx{recipe}[6][2={},4={0.48},5={0.48},6={},usedefault]{%
	\phantomsection
	\begin{center}
		{\LARGE\textbf{#1}} \\
		#2
	\end{center}
	\index{#1}\label{#1}\addcontentsline{toc}{subsection}{#1}
	\ifthenelse{\equal{#6}{}}{%
		\begin{center}
			\begin{tabular}{@{}ll@{}}\toprule
				\ingredientname{}&\quantityname{}\\\midrule
				#3\\\bottomrule
			\end{tabular}
		\end{center}
	}{%
		\vspace{0.5cm}
		\hspace*{\fill}
		\begin{minipage}{#4\columnwidth}
			\begin{center}
				\begin{tabular}{@{}ll@{}}\toprule
					\ingredientname{}&\quantityname{}\\\midrule
					#3\\\bottomrule
				\end{tabular}
			\end{center}
		\end{minipage}
		\hspace*{\fill}
		\begin{minipage}{#5\columnwidth}
			\centering	\includegraphics[width=\columnwidth]{#6}
		\end{minipage}
		\hspace*{\fill}
		\vspace{0.3cm}
	}
	\setcounter{step}{0}
}{%
\vspace{0.3cm}
\[\star\star\star\]
\clearpage
}

\renewcommand{\headrulewidth}{0pt}
\fancyhead[RE,LO]{}
\fancyhead[RO,LE]{}
\cfoot{}
\fancyfoot[RO,LE]{\thepage}
\pagestyle{fancy}
\thispagestyle{fancy}

\begin{document}

\begin{titlepage}

\begin{center}
{\Huge\bfseries Recipe book}

\vspace*{1.5cm}%

\end{center}

\vspace*{0.5cm}
\renewcommand{\contentsname}{Recipe list}
\tableofcontents
\end{titlepage}

\partie{\textit{Plats en sauce}}


\begin{recipe}{Estouffade}[for 6 people]{%
Beef stew& 1 - $\nicefrac{1}{2}$ lb\\
Red wine - strong (Syrah, Merlot)& 1 bottle\\
Large Onions& 3\\
Butter& \\
(Optional) Lardons& $\nicefrac{1}{2}$lbs\\
Garlic& 3 cloves\\
Flour& 2 tbsp\\
Carrots - can be replaced in part by Parsnips& 4 medium-sized\\
Beefstock& 2 cubes\\
Bouquet Garni (sage, thyme, rosemary)& 1\\
Cloves& 6\\
Mushrooms& 2 lb\\
(Optional) Roux
}
\step If you have lardons, cook them in butter until golden, then remove them, leaving the fat. Otherwise, continue.
\step Cut 2 onions in quarters, spike the $3^{r}$ with the cloves
\step Cut carrots in cylinders 1$\nicefrac{1}{2}$ - 2 inches long
\step Dissolve the 2 cubes of beefstock in hot water (.75 cl)
\step Melt butter and sear beefstew and the cut onions until browned
\step Add pressed or grated garlic
\step Add flour and sear till golden (\textit{singer})
\step Add Carrots (and optionnally Parsnips)
\step Add beefstock and red wine (\textit{mouiller}), add water till 90\% of the height of the solids
\step Add Bouquet Garni and clove-spiked onion
\step Add lardons
\step Simmer for 2$\nicefrac{1}{2}$ - 3 hours (low heat). Taste after 1$\nicefrac{1}{2}$ hours and adjust seasonning (salt, pepper, water).
\step Towards the end, sautée the sliced mushrooms in butter and add to pot
\step (Optional) To thicken the sauce, make a Roux, then mix with 3 or 4 ladles of sauce before adding to the pot.
\step Taste and adjust seasonning. Serve hot on steamed rice or potatoes.

\end{recipe}


\begin{recipe}{Estouffade à la Provençale}[for 6 people]{%
Beef stew& 1 - $\nicefrac{1}{2}$ lb\\
White wine - strong - Sauvignon Blanc/Muscadet& 1 bottle\\
Large Onions& 3\\
Olive oil& \\
(Optional) Lardons& $\nicefrac{1}{2}$lbs\\
Garlic& 3 cloves\\
Flour& 2 tbsp\\
Carrots - can be replaced in part by Parsnips& 4 medium-sized\\
Chickenstock& 2 cubes\\
Bouquet Garni (sage, thyme, rosemary)& 1\\
Cloves& 6\\
Mushrooms& 2lb\\
Tomatoes& 1lb\\
Pitted green Olives& $\nicefrac{1}{2} lb$
}

\step Same as Estouffade until step 12 but replace Red Wine with White Wine and butter with olive oil.
\step Cut tomatoes in half. Remove liquid \& seeds. Cut flesh in small cubes.
\step Right after adding the liquid (chickenbroth and white wine) add tomato cubes.
\step 30min before end of simmering, add olives.
\step Taste and adjust seasonning. Serve hot on steamed rice or potatoes.

\end{recipe}

\partie{Preparations}

\begin{recipe}{Roux}{
	Butter& 3 tbsp\\
	Flour& Same weight as butter
}
\step Melt butter in pan
\step Add flour progressively
\step Mix and cook to desired color. The darker the nuttier, the lighter the more thickness it will bring to sauce.
\end{recipe}


\end{document}
